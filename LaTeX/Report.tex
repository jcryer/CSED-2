\documentclass[12pt]{report}
%\usepackage[utf8]{inputenc}
\usepackage{titlesec}
\usepackage[hyphens]{url}

%Removes 'Chapter x' at the start of each section
\titleformat{\chapter}[display] {\normalfont\bfseries}{}{0pt}{\Large}

\title{CSED Coursework 2}
\author{
	Adam George
	\and
	Eddy Dunton
	\and
	Fraser Dwyer
	\and	
	Lance Garcia
	\and
	Sam Davidson
	\and
	Sam Freeman
	\and
	Sandra Comradi
	\and
	Teodora Dinca
}	
\date{Semester 2,  2019-2020}

\begin{document}

\begin{titlepage}
\maketitle
\end{titlepage}

\tableofcontents

\chapter{Introduction}

\section{Personal Informatics - FD}

Personal informatics is a philosophy in which technology can aid the daily lives of people by collecting and processing data. This data can be monitored or manually entered in order to be collected. Personal Informatics tools are used for a variety of reasons, for example, to be therapeutic, or to change or improve one’s behaviour or psychology. Essentially, Personal Informatics tools accumulate someone’s data and feeds it back to the user through some form of data representation. For example, self-monitoring calorie consumption can help to keep track of diets, using graphs and other visual aids to demonstrate trends in the data. Personal Informatics can aid with “…internal states (such as mood or glucose level in the blood) or indicators of performance (such as the kilometres run).”  \cite{Rapp2014a} . Evidently, PI covers a vast amount of areas in our day to day lives, hence why PI tracking tools are so helpful.

With ever growing technological advancements, being able to use more and more PI trackers is becoming much easier. With technologies such as smart watches, sensors in our phones, Fitbits, etc… we are constantly surrounded by computers capable of PI tracking. Furthermore, “the pervasiveness of self-tracking in modern smartphones foreshadows an era where Personal Informatics will likely become ubiquitous making personal data available with minimal burden, easing the process of self-monitoring.” This means that it’ll become effortless for people to have PI capable tracking technologies and hence more people will be able to take advantage of such apps and be able to benefit from them.\cite{Rapp2016}

However, PI tracking tools aren’t without their flaws. Many PI tracking tools lack helpful suggestions and consequently don’t always give users the helpful insight they were after. A common issue is the “excess of abstract visualisation in the apps” \cite{Rapp2016} which consequently can lead to users losing interest in PI tracking apps. Moreover, “we believe that current PI tools are not yet designed with enough understanding of these users’ needs, desires and problems that they may encounter.” \cite{Rapp2016} This implies that despite the overload of information we can be provide for these apps, the information given in return isn’t as useful as it should be.


\section{Proposed Solution - SD}

Since the problem domain is personal informatics, we have decided to research how music can affect a listeners' mood and perspective based mainly on the type of music they are listening to, whether that is upbeat music or it’s a more calming sound, etc. We will also be looking into how different genres will impact on users, and we will take into consideration the time of day as well. Our application will analyse this data and return the mood that a user would generally feel while listening to it. We will be collecting this data from music streaming app Spotify. We would like to have the framework to include different music streaming services such as Apple Music, Deezer, etc.

We know of features which are similar to what we aim to create. For example, Spotify Wrapped, which started as a simple microsite in 2015, showing users how they engaged with the service \cite{Swant}. In 2018, they introduced a personalised feature based off the original in 2015. This personalised feature shows their most listened-to artists, albums, songs, playlists and features from across the year for all users \cite{Somerville}. This has required Spotify to start taking users listening data in order for them to create their annual list for each user. This helps us as the Spotify API which we are going to end up using has a lot of features which we can implement into our own work \cite{WebAPI}. We can use this API to look at the danceability, energy, liveliness, loudness, etc. These will be taken into account when analysing the data.

For this project we will primarily be looking into the Spotify API only in order to show the main features and the capability of our application, however, we will generalise a lot of the code, this will greatly increase the extensibility of the application. This makes it so we able to add other APIs such as Apple Music and Deezer. Doing this, our application would be able to be used by more than one music streaming apps, broadening our target audience, making it accessible to a larger number of potential users. This will maximise the amount of people we can potentially help with our app.

There have been a number of articles related to studies conducted linking people’s moods to upbeat or sad songs. According to a study by Psych Central, upbeat music more than likely raises the mood and perception of a listener \cite{Nauert2018}. Upbeat music tends to be more happier. It has also been established that people listen to sad music are often sad and are looking for comfort, pleasure or pain \cite{Eerola2016}. All this type of information will be very useful in determining the mood and perspective of a listener. However, there are other things we must include like what makes exactly makes a song a happy one or a sad one, and whether a genre has an impact on being happy or sad.


\chapter{Requirements}

\section{Gathering - FD}

In order to make our system as useful as possible, we decided to collect our requirements from a wide range of varying resources. Primarily, we researched heavily into the area surrounding Personal Informatics and based a large portion of our requirements off this. By gathering a lot of requirements from articles such as those that have been peer reviewed allows us to have a better understanding about the sorts of things that our final system should and shouldn’t be doing.

Moreover, we held a focus group. During this focus group, several requirements that we had missed were made apparent. By having a semi-structured conversation with potential stake holders, we were able to see roughly what a large proportion of people would want from a system like the one that we are developing. In addition to this, the focus group allowed us to quickly and efficiently gather data from people who most greatly reflect the audience we are aiming for. This meant that we had more time to work on constructing the system as opposed to researching for more requirements.

Finally, we also sent out a questionnaire. We received X RESPONSES, from which we were able to gather additional requirements. From the focus group, a mass of information was yielded which showed us a lot of perspectives we hadn’t necessarily put a lot of thought into. It also provided many niche requirements which if we have time to get to would really enrich our system and take it a step above where we were aiming before.

\section{Priorities - FD}

Upon reviewing all the data we had gathered to compile our list of requirements, we labelled all of our requirements with priorities and dependencies. The dependencies really highlighted how crucial some requirements were and which of our requirements were more of the additional extras which would be nice to include if time allows. We considered the necessity of each requirement and its dependencies for the priority of any given requirement.

\section{Conflicts - FD}

When we had created our list of requirements, we were careful to avoid any conflicting requirements. These could have arisen from varying opinions from different stake holders. In order to avoid including conflicting requirements, we considered all the data that we had collected and then prioritised the one with the highest interest. Finally, we split all of our requirements into a structured indexed list to help illustrate dependencies and also to avoid repetition or conflict between requirements.

\section{Functional}

\begin{table}[]
\begin{tabular}{lll}
\hline
\multicolumn{3}{|l|}{1. Requirement Name - Server} \\ \hline
1.1 & Send data to client when requested & Author: Sam Davidson \\
\multicolumn{2}{l}{\begin{tabular}[c]{@{}l@{}}The web app must attempt to send data to the client when requested.\\ This data could be specified so it needs to be able to handle the\\ specific request. The web app will then deal with the data it is given.\end{tabular}} & \begin{tabular}[c]{@{}l@{}}Priority: HIGH\\ Dependencies: 2.1\end{tabular} \\ \hline
\multicolumn{3}{|l|}{2. Requirement Name - Data Collection} \\ \hline
2.1 & Get data from Spotify & Author: Joseph Cryer \\
\multicolumn{2}{l}{\begin{tabular}[c]{@{}l@{}}The server must be able to request data from Spotify’s public web API for a user. This can only happen once this user has connected to Spotify through our system successfully. \\ 	\\ Our system must send a message to the Spotify web API containing the authorisation token for that user (see 'Connect to Spotify'). On success, Spotify's web API will return an access token that can then be used to make requests to the API for a certain amount of time.\end{tabular}} & \begin{tabular}[c]{@{}l@{}}Priority: HIGH\\ Dependencies: None\end{tabular} \\ \hline
\multicolumn{3}{|l|}{3. Requirement Name - Data Storage and Processing} \\ \hline
3.1 & Store data from Spotify for each user & Author: Teodora Dinca \\
\multicolumn{2}{l}{The server must process the data received from the Spotify API and take all the data that will be used and put it into a database in an efficient intermediate format keeping only the data we need and not storing any redundant data which offers little to no value to our client.} & \begin{tabular}[c]{@{}l@{}}Priority: HIGH\\ Dependencies: 2.1\end{tabular} \\
3.2 & Recommend songs based on trends & Author: Fraser Dwyer \\
\multicolumn{2}{l}{The system should be able to recommend songs to the user based off of mood patterns or based off of genre that the user likes to listen to.} & \begin{tabular}[c]{@{}l@{}}Priority: MEDIUM\\ Dependencies: None\end{tabular} \\ \hline
\end{tabular}
\end{table}

%Loads bibliography
\bibliography{ref}{}
\bibliographystyle{bathx}

\end{document}